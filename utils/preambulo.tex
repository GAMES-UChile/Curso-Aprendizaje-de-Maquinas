%!TEX root = ../notas_de_clase.tex

%preamble

\usepackage[letterpaper, portrait, margin=0.8in]{geometry}
\usepackage{xcolor}
\usepackage{hyperref}
\hypersetup{
    colorlinks = false,
    linkbordercolor = {white}, 
    urlcolor = {white},
    urlbordercolor = {white}
}

%language
\usepackage[spanish,es-nodecimaldot]{babel}
\usepackage[utf8x]{inputenc}
\usepackage{apacite}

%packages
\usepackage[Algoritmo]{algorithm}
\usepackage{algorithmicx}
\usepackage[noend]{algpseudocode}
\usepackage{mathtools}
\setlength {\marginparwidth}{2cm}
\usepackage{todonotes}
\usepackage{amsbsy}
\usepackage{amssymb}
\usepackage{amsmath,bm}
\usepackage{dsfont}
\usepackage{graphicx}

\usepackage{fancybox}
\usepackage{booktabs}
\usepackage{caption}
\usepackage{float}
\usepackage{amssymb}
\usepackage{amstext}
\usepackage{bm}
\usepackage{wrapfig}
\usepackage{subcaption}
\usepackage[export]{adjustbox} %formatting

%caption para figuras
\captionsetup[figure]{width=.8\linewidth, font=small,labelfont={bf},name={Fig.},labelsep=period}
\captionsetup[table]{width=.8\linewidth,font=small,labelfont={bf},name={Tabla},labelsep=period}


\ifx\byn\undefined
    \definecolor{my_blue}{HTML}{C2D5FF}
    \definecolor{my_red}{HTML}{FFC2C2}
    \definecolor{my_yellow}{HTML}{FFFFE0}
\else
    \definecolor{my_blue}{HTML}{FFFFFF}
    \definecolor{my_red}{HTML}{FFFFFF}
    \definecolor{my_yellow}{HTML}{FFFFFF}
\fi

\usepackage[framemethod=TikZ]{mdframed}

\mdfdefinestyle{discusion}{%
    %linecolor=black,
    %outerlinewidth=0pt,
    roundcorner=0pt,
    innertopmargin=5pt,
    innerbottommargin=5pt,
    innerrightmargin=20pt,
    innerleftmargin=20pt,
    backgroundcolor=my_blue}

\colorlet{Green}{green!90}

\mdfdefinestyle{ejemplo}{%
    %linecolor=black,
    %outerlinewidth=0pt,
    roundcorner=0pt,
    innertopmargin=5pt,
    innerbottommargin=5pt,
    innerrightmargin=20pt,
    innerleftmargin=20pt,
    backgroundcolor=my_yellow}

\mdfdefinestyle{pendiente}{%
    style = discusion, 
    backgroundcolor=my_red}

\RequirePackage{url}

%definitions
\def\td{{\text d}}
\def\cN{{\mathcal N}}
\def\cX{{\mathcal X}} 
\def\cC{{\mathcal C}} 
\def\N{{\mathbb N}}
\def\d{{\text d}}
\def\datos{{\mathcal D}}
\def\eye{{\mathbb I}}
\def\ssum{{\scriptstyle\sum}}
\def\bepsilon{{\bm \epsilon}}
\def\tx{\tilde{x}}
\def\tX{\tilde{X}}
\def\thetaMAP{\theta_\text{MAP}}
\newcommand{\gp}{\ensuremath{\mathcal{GP}}}
\newcommand{\pr}{\ensuremath{\mathbb{P}}}
\newcommand{\x}{\ensuremath{\mathbf{x}}}
\newcommand{\z}{\ensuremath{\mathbf{z}}}
\newcommand{\cvector}{\ensuremath{\mathbf{c}}}
\newcommand{\e}{\ensuremath{\mathbf{e}}}
\newcommand{\y}{\ensuremath{\mathbf{y}}}
\newcommand{\R}{\mathbb{R}}
\newcommand{\norm}[1]{\left\lVert#1\right\rVert}

\DeclareMathOperator*{\argmax}{arg\,max}
\DeclareMathOperator*{\argmin}{arg\,min}
\DeclareMathOperator{\E}{\mathbb{E}}
\DeclareMathOperator{\V}{\mathbb{V}}
\DeclareMathOperator{\KL}{\text{KL}}
\DeclareMathOperator{\MVN}{\text{MVN}}
\newcommand\deq{\stackrel{\mathclap{\normalfont\mbox{\tiny def}}}{=}}
\newcommand{\trace}[1]{\text{Tr} \left[#1\right]}


\usepackage{amsthm}
\newtheorem{theorem}{Teorema}[section]
\newtheorem{corollary}{Corollary}[theorem]
\newtheorem{lemma}{Proposición}[theorem]
\theoremstyle{definition}
\newtheorem{definition}{Definición}[section]
\newtheorem{remark}{Observación}[section]
\renewcommand{\qedsymbol}{$\blacksquare$}

%listing package para código
\usepackage{listings}
\usepackage{xcolor}
 
\definecolor{codegreen}{rgb}{0,0.6,0}
\definecolor{codegray}{rgb}{0.5,0.5,0.5}
\definecolor{codepurple}{rgb}{0.58,0,0.82}
\definecolor{backcolour}{rgb}{0.95,0.95,0.92}
 
\lstdefinestyle{mystyle}{
    xleftmargin=0.15\textwidth,
    linewidth=0.8\textwidth,
    backgroundcolor=\color{backcolour},   
    commentstyle=\color{codegreen},
    keywordstyle=\color{magenta},
    numberstyle=\tiny\color{codegray},
    stringstyle=\color{codepurple},
    basicstyle=\ttfamily\footnotesize,
    breakatwhitespace=true,         
    breaklines=true,                 
    captionpos=b,                    
    keepspaces=true,                 
    numbers=left,                    
    numbersep=5pt,                  
    showspaces=false,                
    showstringspaces=false,
    showtabs=false,                  
    tabsize=2
}
 
\lstset{style=mystyle}

\numberwithin{equation}{section}
