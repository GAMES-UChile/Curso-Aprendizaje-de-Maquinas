%!TEX root = ../notas_de_clase.tex
\newpage
\section*{Prefacio}
Este apunte es una versión extendida y detallada de las notas de clase utilizadas en el curso \textsc{MDS7104: Aprendizaje de Máquinas} (ex MA5203 y MA5204) dictado anualmente en el \emph{Master of Data Science} de la Facultad de Ciencias Físicas y Matemáticas de la  Universidad de Chile entre 2016 y 2024. El objetivo principal de este apunte es presentar material autocontenido y original de las temáticas vistas en el curso tanto para apoyar su realización como para estudio personal de quien lo requiera. Debido a que los contenidos del curso van variando año a año, el apunte está en constante modificación, por esta razón hay secciones de este documento que pueden estar incompletas en cuanto a formato, figuras o contenidos. Sin embargo, creo que el hacer disponible este apunte en desarrollo puede ser un aporte para los alumnos del curso MDS7104 como a la comunidad en general.

El desarrollo de este apunte solo ha sido posible gracias a la contribución de varios integrantes del cuerpo académico de los cursos \textsc{MA5203: Aprendizaje de Máquinas Probabilístico} (2016-2018), \textsc{MA5309: Aprendizaje de Máquinas Avanzado} (2016, 2018, 2020 y 2022), \textsc{MA5204: Aprendizaje de Máquinas} (2019-2021) y \textsc{MDS7104: Aprendizaje de Máquinas} (2022-2024). Me gustaría reconocer la indispensable contribución de ayudantes y participantes de estos cursos, tanto en el desarrollo del curso mismo, ideas de tareas y clases auxiliares, producción de figuras, ejemplos, y mucho más. En orden de \emph{aparición}, gracias: 
Gonzalo Ríos, 
Camilo Carvajal,
Cristóbal Silva, 
Alejandro Cuevas, 
Alejandro Veragua, 
Cristóbal Valenzuela, 
Mauricio Campos, 
Lerko Araya, 
Nicolás Aramayo, 
Mauricio Araneda, 
Mauricio Romero, 
Luis Muñoz, 
Jou-Hui Ho, 
Diego Garrido, 
José Díaz, 
Francisco Vásquez, 
Fernando Fetis, 
Nelson Moreno, 
Arie Wortsman, 
Víctor Faraggi, 
David Molina, 
Tomás Valencia, 
Alonso Vargas, 
Fernando F\^etis, 
Augusto Tagle, 
Juan Cuevas.
Sin la participación de todos ustedes, el éxito de los cursos mencionados y la composición de este apunte no habría sido posible.





\bigskip
\begin{flushright}
  Felipe Tobar\par
  7 de abril de 2024\par
  Santiago, Chile
\end{flushright}
\newpage

