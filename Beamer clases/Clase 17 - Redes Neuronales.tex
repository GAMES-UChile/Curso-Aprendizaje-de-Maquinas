\documentclass[9pt]{beamer}
%\include{config/commands}
%
% Choose how your presentation looks.
%
% For more themes, color themes and font themes, see:
% http://deic.uab.es/~iblanes/beamer_gallery/index_by_theme.html
%
%!TEX root = ../notas_de_clase.tex

%preamble

%language
\usepackage[spanish,es-nodecimaldot]{babel}
\usepackage[utf8]{inputenc}
\usepackage{apacite}
\usepackage[absolute,overlay]{textpos}

%packages
\usepackage[Algoritmo]{algorithm}
\usepackage{algorithmicx}
\usepackage[noend]{algpseudocode}
\usepackage{mathtools}
\setlength {\marginparwidth}{2cm}
\usepackage{todonotes}
\usepackage{amsbsy}
\usepackage{amssymb}
\usepackage{amsmath,bm}
\usepackage{dsfont}

\usepackage{comment}

\usepackage{xcolor}
\providecommand{\sred}[1]{\textcolor{red}{#1}}
\providecommand{\sblue}[1]{\textcolor{blue}{#1}}
\providecommand{\red}[1]{\textcolor{red}{\text{#1}}}
\providecommand{\blue}[1]{\textcolor{blue}{\text{#1}}}
\providecommand{\redb}[1]{\textcolor{red}{\textbf{#1}}}
\providecommand{\blueb}[1]{\textcolor{blue}{\textbf{#1}}}
\usepackage{graphicx}
\usepackage{fancybox}
\usepackage{booktabs}
\usepackage{caption}
\usepackage{float}
%\usepackage[longend,ruled,algochapter,linesnumbered,lined,boxed,commentsnumbered,spanish]{algorithm2e}
%\usepackage[algo2e]{algorithm2e}
\usepackage{amssymb}
\usepackage{amstext}
\usepackage{bm}
\usepackage{wrapfig}
\usepackage{subcaption} % para_unsupervised_chapter

%formatting

\usepackage[export]{adjustbox}

%caption para figuras
\captionsetup[figure]{width=.8\linewidth, font=small,labelfont={bf},name={Fig.},labelsep=period}
\captionsetup[table]{width=.8\linewidth,font=small,labelfont={bf},name={Tabla},labelsep=period}



\ifx\byn\undefined
    \definecolor{my_blue}{HTML}{C2D5FF}
    \definecolor{my_red}{HTML}{FFC2C2}
    \definecolor{my_yellow}{HTML}{FFFFE0}
\else
    \definecolor{my_blue}{HTML}{FFFFFF}
    \definecolor{my_red}{HTML}{FFFFFF}
    \definecolor{my_yellow}{HTML}{FFFFFF}
\fi


\usepackage[framemethod=TikZ]{mdframed}
\mdfdefinestyle{discusion}{%
    %linecolor=black,
    %outerlinewidth=0pt,
    roundcorner=0pt,
    innertopmargin=5pt,
    innerbottommargin=5pt,
    innerrightmargin=20pt,
    innerleftmargin=20pt,
    backgroundcolor=my_blue}

\colorlet{Green}{green!90}


\mdfdefinestyle{ejemplo}{%
    %linecolor=black,
    %outerlinewidth=0pt,
    roundcorner=0pt,
    innertopmargin=5pt,
    innerbottommargin=5pt,
    innerrightmargin=20pt,
    innerleftmargin=20pt,
    backgroundcolor=my_yellow}


\mdfdefinestyle{pendiente}{%
    style = discusion, 
    backgroundcolor=my_red}


\RequirePackage{url}



%definitions
\def\td{{\text d}}
\def\cN{{\mathcal N}}
\def\cX{{\mathcal X}} 
\def\cC{{\mathcal C}} 
\def\N{{\mathbb N}}
\def\d{{\text d}}
\def\datos{{\mathcal D}}
\def\eye{{\mathbb I}}
\def\ssum{{\scriptstyle\sum}}
\def\bepsilon{{\bm \epsilon}}
\def\tx{\tilde{x}}
\def\tX{\tilde{X}}
\def\thetaMAP{\theta_\text{MAP}}
\newcommand{\gp}{\ensuremath{\mathcal{GP}}}
\newcommand{\pr}{\ensuremath{\mathbb{P}}}
\newcommand{\x}{\ensuremath{\mathbf{x}}}
\newcommand{\z}{\ensuremath{\mathbf{z}}}
\newcommand{\cvector}{\ensuremath{\mathbf{c}}}
\newcommand{\e}{\ensuremath{\mathbf{e}}}
\newcommand{\y}{\ensuremath{\mathbf{y}}}
\newcommand{\bx}{\ensuremath{\textcolor{blue}{X}}}
\newcommand{\by}{\ensuremath{\textcolor{blue}{Y}}}
\newcommand{\rx}{\ensuremath{\textcolor{red}{X_*}}}

\newcommand{\R}{\mathbb{R}}
\newcommand{\norm}[1]{\left\lVert#1\right\rVert}




\DeclareMathOperator*{\argmax}{arg\,max}
\DeclareMathOperator*{\argmin}{arg\,min}
\DeclareMathOperator{\E}{\mathbb{E}}
\DeclareMathOperator{\V}{\mathbb{V}}
\DeclareMathOperator{\KL}{\text{KL}}
\DeclareMathOperator{\MVN}{\text{MVN}}
\newcommand\deq{\stackrel{\mathclap{\normalfont\mbox{\tiny def}}}{=}}
%\newcommand{\E}[1]{\mathbb E \left[#1\right]}
\newcommand{\trace}[1]{\text{Tr} \left[#1\right]}


\usepackage{amsthm}

%-------------------------------------------
% Newtheorem
%-------------------------------------------
\newtheorem{axioma}{\textcolor{red}{Axioma}}
\newtheorem{definicion}{Definición}
\newtheorem*{notacion}{Notación}
\newtheorem{teorema}{Teorema}
\newtheorem{corolario}{Corolario}
\newtheorem{lema}{Lema}
\newtheorem{lemaZ}{\textcolor{red}{Lema}}
\newtheorem{propiedad}{Propiedad:}
\newtheorem{proposicion}{Proposición:}
\newtheorem*{observacion}{Observación}
\newtheorem*{comentario}{Comentario}
\newtheorem*{ejemplo}{Ejemplo}
\newtheorem*{resultado}{Resultado}
\newtheorem*{propuesto}{Ejercicio propuesto}
\newtheorem*{demostracion}{Demostración} % No se usa, usar \begin{proof}\end{proof} que son por default.

%listing paackage para código
\usepackage{listings}
\usepackage{xcolor}
 
\definecolor{codegreen}{rgb}{0,0.6,0}
\definecolor{codegray}{rgb}{0.5,0.5,0.5}
\definecolor{codepurple}{rgb}{0.58,0,0.82}
\definecolor{backcolour}{rgb}{0.95,0.95,0.92}
 
\lstdefinestyle{mystyle}{
    xleftmargin=0.15\textwidth,
    linewidth=0.8\textwidth,
    backgroundcolor=\color{backcolour},   
    commentstyle=\color{codegreen},
    keywordstyle=\color{magenta},
    numberstyle=\tiny\color{codegray},
    stringstyle=\color{codepurple},
    basicstyle=\ttfamily\footnotesize,
    breakatwhitespace=true,         
    breaklines=true,                 
    captionpos=b,                    
    keepspaces=true,                 
    numbers=left,                    
    numbersep=5pt,                  
    showspaces=false,                
    showstringspaces=false,
    showtabs=false,                  
    tabsize=2
}
 
\lstset{style=mystyle}

\numberwithin{equation}{section}

\usetheme{simple}

% \usepackage[T1]{fontenc} %Este es el mismo utilizado en el tex de clases salvo modificaciones
\title{Clase 17 - Redes Neuronales II}
\subtitle{Aprendizaje de Máquinas - MA5204}
\date{\today}
\author{Felipe Tobar} 
\titlegraphic{
\begin{figure}[htp] 
    \centering
        \includegraphics[width=0.15\textwidth]{../img/Uchile.pdf}%
    %\hspace{2em}% 
    %    \includegraphics[width=0.15\textwidth]{img/CMM.pdf}%
         
\end{figure}
}
\institute{Department of Mathematical Engineering \&\\ Center for Mathematical Modelling\\Universidad de Chile}
%-------------------------------------------
% Inicio del documento, no tocar la config. de portada
%-------------------------------------------
\begin{document}
% Portada
\begin{frame}
  \titlepage
\end{frame}
% Tabla de contenidos
\begin{frame}{Contenido}
  \tableofcontents
  
\end{frame}
% Portada de seccion
\AtBeginSection[]{
  \begin{frame}
  \vfill
  \centering
  \begin{beamercolorbox}[sep=8pt,center,shadow=true,rounded=true]{title}
    \usebeamerfont{title}\insertsectionhead\par
  \end{beamercolorbox}
  \vfill
  \end{frame}}

%tcolorbox
%-------------------------------------------
% Contenido
%-------------------------------------------
% Nueva sección, o capitulo, tiene diapo de portada propio, basta con ponerla fuera del entorno frame
\section{Entrenamiento de una Red Neuronal}

\begin{frame}{Forward propagation}



\end{frame}

\begin{frame}{Backpropagation - Introducción}

\end{frame}

\begin{frame}{Backpropagation - Preliminares}

\begin{itemize}
  \item Utilicemos la notación para el paso intermedio antes de aplicar la función de activación como $u = hW + b$ \pause

  \item Supongamos además que el entrenamiento de la red se realiza con un paquete de datos (batch), es decir $(x^d)_{d=1}^N$  conjunto de $N$ inputs, luego $u_{dj}^k$ corresponde al valor de $u$ para el input $d$ en el nodo j para la capa $k$ \pause
 
 \item Se verá el algoritmo de backpropagation utilizando funciones de activación sigmoide y para función de error MSE. \pause

\end{itemize} 
\[
J(X , \theta) = \frac{1}{2N}\sum_{d=1}^N(f(X,\theta)_d-y_d)^2
\]
Nuestro objetivo será actualizar todos los valores $w_{ij}^k$ (peso del nodo $i$ al nodo $j$ en la capa $k$). Es decir, para utilizar el método del descenso de gradiente es necesario calcular $\frac{\partial J(X , \theta) }{\partial w_{ij}^k}$ notemos que \pause
\[
\frac{\partial J(X , \theta) }{\partial w_{ij}^k} = \frac{1}{N}\sum_{d=1}^N \frac{\partial}{\partial w_{ij}^k} \left ( \frac{1}{2}(\hat{y}_d-y_d)^2 \right) = \frac{1}{N}\sum_{i=1}^N \frac{\partial J_d}{\partial w_{ij}^k}
\]
Veamos esto último
\end{frame}

\begin{frame}{Backpropagation - Preliminares}

Utilizando la regla de la cadena 
\[
\frac{\partial J_d}{\partial w_{ij}^k} = \frac{\partial J_d}{\partial u_{dj}^k}\frac{\partial u_{dj}^k}{\partial w_{ij}^k}
\] \pause
La expresión $\frac{\partial J_d}{\partial u_{dj}^k}$ corresponde a un término de \textbf{error} y lo denotaremos \
\[
\delta_{dj}^k \equiv \frac{\partial J_d}{\partial u_{dj}^k}
\] \pause
Mientras que para el otro término tenemos que 
\[
\frac{\partial u_{dj}^k}{\partial w_{ij}^k} = \frac{\partial}{\partial w_{ij}^k} \left ( \sum_{a = 1}^{k_k}w_{aj}^kh_{da}^{k-1} + b_j^k \right) = h_{di}^{k-1}
\] \pause
y así 
\[
\frac{\partial J_d}{\partial w_{ij}^k} = \delta_{dj}^k  h_{di}^{k-1}
\] \pause
El gradiente final, será la suma de todos los gradientes y que expresaremos en su forma matricial
\[\frac{\partial J}{\partial w_{ij}^k} = \sum_{d=1}^N \delta_{dj}^k  h_{di}^{k-1}  \Rightarrow  \frac{\partial J}{\partial W^k} = (h^{k-1})^T @ \hspace{0.1cm} \delta^{k} 
\]
Omitiremos la constante $1/N$ hasta el final del algoritmo.
\end{frame}
\begin{frame}{Backpropagation - Capas ocultas}
Nuevamente, de la regla de la cadena \pause

\[
\delta_{dj}^k = \frac{\partial J_d}{\partial u_{dj}^k} = \sum_{a=1}^{k_{k+1}} \frac{\partial J_d}{\partial u_{da}^{k+1}} \frac{\partial u_{da}^{k+1}}{\partial u_{dj}^k} = \sum_{a=1}^{k_{k+1}} \delta_{da}^{k+1} \frac{\partial u_{da}^{k+1}}{\partial u_{dj}^k} 
\] \pause 
no es difícil ver que

\[
\frac{\partial u_{da}^{k+1}}{\partial u_{dj}^k} = w_{ja}^{k+1}f'(u_{dj}^k)  \Rightarrow  \delta_{dj}^k = f'(u_{dj}^k)\sum_{a=1}^{k_{k+1}}w_{ja}^{k+1}\delta_{da}^{k+1}
\]
\pause
Hemos encontrado una expresión para calcular el gradiente en una capa $k$ en base al gradiente de la siguiente capa $k+1$ (De aquí viene el nombre backward propagation).
\newline \pause

Lo anterior en su forma matricial resulta en 
\[
\delta^{k} = f'(u^k) * \left ( \delta^{k+1} @ \hspace{0.1cm} (W^{k+1})^T \right )
\] \pause
Lo único que queda para presentar el algoritmo final es calcular los gradientes en la última capa (capa de output)

\end{frame}

\begin{frame}{Backpropagation - Capa de output}
Estamos suponiendo un problema de regresión por lo que el output será de una sola salida
\[
\delta_{d1}^l = \frac{\partial J_d}{\partial u_{d1}^k} = (\hat{y}_d-y_d) (\hat{y}_d)' 
\] 
Además, la función de activación en el output será lineal y por tanto $(\hat{y}_d)' = 1$, finalmente el término de normalización $N$ se agrega en este paso.  La forma matricial queda en 
\[
\delta_{1}^l = \frac{1}{N}(\hat{y}-y)
\]
\begin{propuesto}
\begin{itemize}
  \item Encontrar una expresión para el gradiente de los bias $(b^{k})_k$
  \item Calcular el gradiente para un problema con unidad de output sigmoidal (problema de clasificación binario) o para un problema con unidad de output softmax (problema de clasificación multiclase) 
\end{itemize}
\end{propuesto}


\end{frame}

\begin{frame}{Backpropagation - Algoritmo}


\end{frame}






%Quitar de comentarios apenas se agregue alguna referencia 
%\bibliography{../capitulos/referencias} %Bibliografía
%\bibliographystyle{apacite}
\end{document} 