\documentclass[handout, 9pt]{beamer}

%!TEX root = ../notas_de_clase.tex

%preamble

%language
\usepackage[spanish,es-nodecimaldot]{babel}
\usepackage[utf8]{inputenc}
\usepackage{apacite}
\usepackage[absolute,overlay]{textpos}

%packages
\usepackage[Algoritmo]{algorithm}
\usepackage{algorithmicx}
\usepackage[noend]{algpseudocode}
\usepackage{mathtools}
\setlength {\marginparwidth}{2cm}
\usepackage{todonotes}
\usepackage{amsbsy}
\usepackage{amssymb}
\usepackage{amsmath,bm}
\usepackage{dsfont}

\usepackage{comment}

\usepackage{xcolor}
\providecommand{\sred}[1]{\textcolor{red}{#1}}
\providecommand{\sblue}[1]{\textcolor{blue}{#1}}
\providecommand{\red}[1]{\textcolor{red}{\text{#1}}}
\providecommand{\blue}[1]{\textcolor{blue}{\text{#1}}}
\providecommand{\redb}[1]{\textcolor{red}{\textbf{#1}}}
\providecommand{\blueb}[1]{\textcolor{blue}{\textbf{#1}}}
\usepackage{graphicx}
\usepackage{fancybox}
\usepackage{booktabs}
\usepackage{caption}
\usepackage{float}
%\usepackage[longend,ruled,algochapter,linesnumbered,lined,boxed,commentsnumbered,spanish]{algorithm2e}
%\usepackage[algo2e]{algorithm2e}
\usepackage{amssymb}
\usepackage{amstext}
\usepackage{bm}
\usepackage{wrapfig}
\usepackage{subcaption} % para_unsupervised_chapter

%formatting

\usepackage[export]{adjustbox}

%caption para figuras
\captionsetup[figure]{width=.8\linewidth, font=small,labelfont={bf},name={Fig.},labelsep=period}
\captionsetup[table]{width=.8\linewidth,font=small,labelfont={bf},name={Tabla},labelsep=period}



\ifx\byn\undefined
    \definecolor{my_blue}{HTML}{C2D5FF}
    \definecolor{my_red}{HTML}{FFC2C2}
    \definecolor{my_yellow}{HTML}{FFFFE0}
\else
    \definecolor{my_blue}{HTML}{FFFFFF}
    \definecolor{my_red}{HTML}{FFFFFF}
    \definecolor{my_yellow}{HTML}{FFFFFF}
\fi


\usepackage[framemethod=TikZ]{mdframed}
\mdfdefinestyle{discusion}{%
    %linecolor=black,
    %outerlinewidth=0pt,
    roundcorner=0pt,
    innertopmargin=5pt,
    innerbottommargin=5pt,
    innerrightmargin=20pt,
    innerleftmargin=20pt,
    backgroundcolor=my_blue}

\colorlet{Green}{green!90}


\mdfdefinestyle{ejemplo}{%
    %linecolor=black,
    %outerlinewidth=0pt,
    roundcorner=0pt,
    innertopmargin=5pt,
    innerbottommargin=5pt,
    innerrightmargin=20pt,
    innerleftmargin=20pt,
    backgroundcolor=my_yellow}


\mdfdefinestyle{pendiente}{%
    style = discusion, 
    backgroundcolor=my_red}


\RequirePackage{url}



%definitions
\def\td{{\text d}}
\def\cN{{\mathcal N}}
\def\cX{{\mathcal X}} 
\def\cC{{\mathcal C}} 
\def\N{{\mathbb N}}
\def\d{{\text d}}
\def\datos{{\mathcal D}}
\def\eye{{\mathbb I}}
\def\ssum{{\scriptstyle\sum}}
\def\bepsilon{{\bm \epsilon}}
\def\tx{\tilde{x}}
\def\tX{\tilde{X}}
\def\thetaMAP{\theta_\text{MAP}}
\newcommand{\gp}{\ensuremath{\mathcal{GP}}}
\newcommand{\pr}{\ensuremath{\mathbb{P}}}
\newcommand{\x}{\ensuremath{\mathbf{x}}}
\newcommand{\z}{\ensuremath{\mathbf{z}}}
\newcommand{\cvector}{\ensuremath{\mathbf{c}}}
\newcommand{\e}{\ensuremath{\mathbf{e}}}
\newcommand{\y}{\ensuremath{\mathbf{y}}}
\newcommand{\bx}{\ensuremath{\textcolor{blue}{X}}}
\newcommand{\by}{\ensuremath{\textcolor{blue}{Y}}}
\newcommand{\rx}{\ensuremath{\textcolor{red}{X_*}}}

\newcommand{\R}{\mathbb{R}}
\newcommand{\norm}[1]{\left\lVert#1\right\rVert}




\DeclareMathOperator*{\argmax}{arg\,max}
\DeclareMathOperator*{\argmin}{arg\,min}
\DeclareMathOperator{\E}{\mathbb{E}}
\DeclareMathOperator{\V}{\mathbb{V}}
\DeclareMathOperator{\KL}{\text{KL}}
\DeclareMathOperator{\MVN}{\text{MVN}}
\newcommand\deq{\stackrel{\mathclap{\normalfont\mbox{\tiny def}}}{=}}
%\newcommand{\E}[1]{\mathbb E \left[#1\right]}
\newcommand{\trace}[1]{\text{Tr} \left[#1\right]}


\usepackage{amsthm}

%-------------------------------------------
% Newtheorem
%-------------------------------------------
\newtheorem{axioma}{\textcolor{red}{Axioma}}
\newtheorem{definicion}{Definición}
\newtheorem*{notacion}{Notación}
\newtheorem{teorema}{Teorema}
\newtheorem{corolario}{Corolario}
\newtheorem{lema}{Lema}
\newtheorem{lemaZ}{\textcolor{red}{Lema}}
\newtheorem{propiedad}{Propiedad:}
\newtheorem{proposicion}{Proposición:}
\newtheorem*{observacion}{Observación}
\newtheorem*{comentario}{Comentario}
\newtheorem*{ejemplo}{Ejemplo}
\newtheorem*{resultado}{Resultado}
\newtheorem*{propuesto}{Ejercicio propuesto}
\newtheorem*{demostracion}{Demostración} % No se usa, usar \begin{proof}\end{proof} que son por default.

%listing paackage para código
\usepackage{listings}
\usepackage{xcolor}
 
\definecolor{codegreen}{rgb}{0,0.6,0}
\definecolor{codegray}{rgb}{0.5,0.5,0.5}
\definecolor{codepurple}{rgb}{0.58,0,0.82}
\definecolor{backcolour}{rgb}{0.95,0.95,0.92}
 
\lstdefinestyle{mystyle}{
    xleftmargin=0.15\textwidth,
    linewidth=0.8\textwidth,
    backgroundcolor=\color{backcolour},   
    commentstyle=\color{codegreen},
    keywordstyle=\color{magenta},
    numberstyle=\tiny\color{codegray},
    stringstyle=\color{codepurple},
    basicstyle=\ttfamily\footnotesize,
    breakatwhitespace=true,         
    breaklines=true,                 
    captionpos=b,                    
    keepspaces=true,                 
    numbers=left,                    
    numbersep=5pt,                  
    showspaces=false,                
    showstringspaces=false,
    showtabs=false,                  
    tabsize=2
}
 
\lstset{style=mystyle}

\numberwithin{equation}{section}

\usetheme{simple}

\title{Clase 17: Redes neuronales (parte 1)}
\subtitle{MA5204 Aprendizaje de Máquinas}
\date{\today}
\author{Felipe Tobar} 
\titlegraphic{
\begin{figure}[htp] 
    \centering
        \includegraphics[width=0.15\textwidth]{../img/Uchile.pdf}% 
\end{figure}
}
\institute{Iniciativa de Datos e Inteligencia Artificial\\Universidad de Chile}

\begin{document}
\begin{frame}
  \titlepage
\end{frame}

\section{Redes Neuronales - Introducción y Arquitectura}
\begin{frame}{Redes Neuronales - Introducción}
Una \textit{Red Neuronal} es un modelo computacional basado en la conexión de múltiples unidades (neuronas) cuyo objetivo* es aproximar una función $x\mapsto y=f(x)$.
\pause

\begin{columns}

  \begin{column}{0.5\textwidth}

  Su principal ventaja radica en la posibilidad de ser entrenarla con datos \emph{crudos}, es decir, de forma agnóstica con respecto de las características relevantes de los datos, con lo que la red \emph{aprende} las características. \pause

  \vspace{0.1cm}
  Los modelos escenciales de redes neuronales se conocen como \textbf{feedforward neural networks}, o \textbf{multilayer perceptrons} (MLPs), Una red feedforward define un mapa $\bm{y}=f(\bm{x}; \bm{\theta})$, donde los parámetros $\bm{\theta}$ se aprenden para tener la \emph{mejor} aproximación posible. \pause

  \vspace{0.1cm}

  Estos modelos se conocen como redes ya que tiene estructura composicional es decir, son típicamente el resultado de composiciones sucesivas de varios tipos de funciones $f^{(1)}$, $f^{(2)}$, $f^{(3)}, \dots$. \pause

  \end{column}
  \begin{column}{0.5\textwidth}
  \begin{figure}[H]
  \centering
  \visible<1->{\includegraphics[scale=.4]{../img/cap7_F1NN1L.png}}
  \caption{Ejemplo de una red neuronal de 1 capa: (\textit{izquierda}). \\  Representación vectorial de las capas. (\textit{derecha})}
  \label{fig:NN}
  \end{figure}

  \end{column}

\end{columns}

\end{frame}

\begin{frame}{Redes Neuronales - El perceptrón}

El \textit{perceptrón} corresponde a la forma más básica de una red neuronal, esta recibe un input numérico $x = (x_i)_{i=1}^n \in \R^n$ y computa la suma ponderada $u = x_1w_1 + x_2w_2 + \dots + w_nx_n + b$ donde $W = (w_i)_{i=1}^n \in \R^n$ corresponden a los \textbf{pesos} (weights) y $b \in \R$ el \textbf{sesgo} (bias). \pause 

Luego, se aplica una \textbf{función de activación} $f$ y se entrega una salida $h = f(u)$. \pause 
\begin{figure}[H]
  \centering
  \visible<3->{\includegraphics[scale=.3]{../img/cap5_activaciones}}
  \caption{Algunos ejemplos de funciones de activación}
\end{figure}

\end{frame}

\begin{frame}{Redes Neuronales - El perceptrón}

Veamos el siguiente ejemplo: \pause

\begin{columns}

  \begin{column}{0.7\textwidth}

  \begin{figure}[H]
    \centering
    \visible<2->{\includegraphics[scale=.3]{../img/cap7_xor}}
    \caption{Red Neuronal de 2 perceptrones para XOR}
  \end{figure}

  \end{column}

  \begin{column}{0.3\textwidth}
  
  \hspace{0.4cm} XOR operator

  \scalebox{1.3}{
  \begin{tabular}{| c  c | c |}
    \hline
    $x_1$ & $x_2$ & $y$ \\ \hline
    0 & 0 & 0 \\
    1 & 0 & 1 \\
    0 & 1 & 1 \\ 
    1 & 1 & 0 \\ \hline
  \end{tabular}
  }

  

  \end{column}

\end{columns} \pause

En su forma matricial 
$$
(h_1 , h_2) = \text{Relu} \left ( \begin{pmatrix}
x_1 & x_2  
\end{pmatrix} \begin{pmatrix}
1 &1 \\ 
 1 & 1
\end{pmatrix} + \begin{pmatrix}
0 & -1 
\end{pmatrix}\right) \hspace{0.5cm} h = \text{Relu}(xW + b)
$$
\pause
$$
y = \begin{pmatrix}
h_1 & h_2 
\end{pmatrix}\begin{pmatrix}
1 \\ 
-2 
\end{pmatrix} + (0) \hspace{0.5cm} y = hU + c
$$
Donde $U$ y $c$ corresponden al peso y bias de la última capa respectivamente. 
\textbf{¿Cómo construir el resto de operadores lógicos?}
\end{frame}

\begin{frame}{Teorema de Aproximación Universal\footnote{(Horniket al., 1989; Cybenko, 1989)} }

Un conjunto de perceptrones conectados unos con otros en secuencia forman arquitecturas capaces de replicar funciones más complejas.  \pause

En el contexto de una red de múltiples perceptrones con una sola capa escondida, se tiene el siguiente resultado \pause

\begin{theorem}[UAT - Ancho arbitrario]

Sea $f^{*}$ una función objetivo Borel Medible (por ejemplo $f^{*}:[0,1]^k \rightarrow [0,1]$ continua con $k \in \N$) y sea f una función de activación no polinomial acotada, entonces para todo $\epsilon > 0$ existen $W,b,U$ definidas como antes tales que 
\[
F(x) = f(xW+b)U, \quad 
|F(x)-f^{*}(x)|<\epsilon \quad \forall x \in Dom(f^{*}) 
\]
\end{theorem} \pause
En palabras simples, con una cantidad suficiente de perceptrones (i.e., con una red lo suficientemente \emph{ancha}), es posible aproximar una función objetivo razonable tan finamente como se desee. 
\end{frame}

\begin{frame}{Redes Neuronales - Arquitectura}

Si bien el UAT establece que es posible aproximar funciones con precisión arbitraria, tenemos dos problemas: 
\begin{itemize}
\item no es claro que los pesos se pueden \emph{aprender}, 
\item no tenemos una cota para la cantidad de neuronas necesarias.
\end{itemize} 
Es por esto que surge la necesidad de arquitecturas más complejas, es decir, con mayor cantidad de capas (mayor profundidad) y no solo neuronas (mayor ancho). \pause

\begin{figure}[H]
  \centering
  \visible<2->{\includegraphics[scale=.3]{../img/cap7_red}}
  \caption{Una red con mayor profundidad}
\end{figure}

\pause 

Utilizaremos la siguiente notación 
\begin{equation}
h^{(k)} = f^{(k)}(h^{(k-1)}W^{(k)} + b^{(k)}) \hspace{0.3cm} \forall k \in \{1 ,\dots, l\}, \quad h^{(0)}=x
\end{equation}
\begin{equation}
\hat{y} = g(h^{(l)}U + c)
\end{equation}

Donde $g$ es la función de la capa de salida y define la $\textbf{unidad de salida}$

\end{frame}

\section{Función de costos y unidades de output}

\begin{frame}{Función de costo}
Principales diferencias entre los modelos lineales antes vistos y una red neuronal:
\begin{itemize}
\item el uso de ciertas funciones de activación hacen que la función de costos \textbf{no sea convexa}
\item SGD no garantiza optimalidad global
\item SGD no garantiza ni siquiera buena solución
\end{itemize}

Interpretando la red neuronal como un \emph{modelo generativo}, y denotando su ley mediante $p(y|\bm{x}; \bm{\theta})$, podemos considerar el costo de entrenamiento de la red como la log-verosimilitud negativa del modelo inducido por la red, y estimar sus parámetros mediante máxima verosimilitud. \pause

La \textbf{función de costo} que utilizaremos entonces será: 

\begin{equation*}
J(\bm{\theta}) = -\log p(y_1,\ldots,y_n|\bm{x}_1,\ldots,\bm{x}_n; \bm{\theta})
\end{equation*}


\end{frame}
     
\begin{frame}{Unidad de salida}

Considerando el procesamiento de la red hasta antes de la capa de salida, es posible interpretar que la red está produciendo \emph{características} que luego son combinadas en la última capa para formar una salida. En este sentido, la elección de la \textbf{unidad de salida} definirá, en gran parte,  la forma que toma la función de costo. \\Algunos ejemplos 
\begin{enumerate}
  \item Salida lineal $\hat{\bm{y}} = h^{(l)}U + c$ \pause

  Útil en problemas de regresión. En particular, $\hat{\bm{y}}$ puede ser la media del modelo propuesto con distrbución Gaussiana condicional $p(\bm{y}|\bm{x}) = \mathcal{N}(\bm{y}|\hat{\bm{y}};\bm{I})$. En dicho caso, la log-verosimilitud será equivalente minimizar el error cuadrático medio. \pause

  \item Unidad de salida sigmoidal $\hat{{y}} = \text{sig}(h^{(l)}U + c)$ \pause

  Se utiliza para escenarios de clasificación binaria, cuyo output es $P(y=1|\bm{x})$ o la probabilidad de pertenecer a la clase 1. \pause

  \item Unidad de output softmax $\hat{{y}} = \text{softmax}(h^{(l)}U + c)$ \pause 

  Es una generalización de la función sigmoidal definida como 
  \begin{equation*}
  \text{softmax}(z)_i = \frac{e^{z_i}}{\sum_{j}e^{z_j}},
  \end{equation*}
  y es útil para el caso de clasificación multiclase. 
  %Se podrían agregar propiedades como estabilidad numérica producto de invarianza a la adición 
\end{enumerate}

\end{frame}



%Quitar de comentarios apenas se agregue alguna referencia 
%\bibliography{../capitulos/referencias} %Bibliografía
%\bibliographystyle{apacite}
\end{document} 