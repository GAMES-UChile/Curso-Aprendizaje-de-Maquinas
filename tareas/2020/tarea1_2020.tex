\documentclass[11pt,letterpaper]{article}
\usepackage[utf8]{inputenc}
\usepackage[letterpaper,includeheadfoot, top=0.5cm, bottom=3.0cm, right=1.5cm, left=1.5cm]{geometry}
\renewcommand{\familydefault}{\sfdefault}
\usepackage{float} % Allows putting an [H] in \begin{figure} to specify the exact location of the figure
\renewcommand{\figurename}{Fig.}
\usepackage{lmodern}% http://ctan.org/pkg/lm
\usepackage{amsmath}
\usepackage{graphicx}
\usepackage{color}
\usepackage{amssymb}
\usepackage{url}
\usepackage{pdfpages}
\usepackage{fancyhdr}
\usepackage{subfig}
\usepackage{listings} %Codigo
\usepackage{selinput}                   % Compatibilidad con acentos
\usepackage[colorlinks = true,
            linkcolor = blue,
            urlcolor  = blue,
            citecolor = blue,
            anchorcolor = blue]{hyperref}
\newcommand{\bR}{\ensuremath{\mathbb{R}}}
\newcommand{\bN}{\ensuremath{\mathbb{N}}}
\newcommand{\bZ}{\ensuremath{\mathbb{Z}}}
\newcommand{\bP}{\ensuremath{\mathbb{P}}}
\newcommand{\bE}{\ensuremath{\mathbb{E}}}
\newcommand{\bD}{\ensuremath{\mathbb{D}}}
\newcommand{\bV}{\ensuremath{\mathbb{V}}}
\newcommand{\cN}{\ensuremath{\mathcal{N}}}
\newcommand{\x}{\ensuremath{\mathbf{x}}}
\newcommand{\m}{\ensuremath{\mathbf{m}}}
\newboolean{pauta}
\setboolean{pauta}{false}
%\usepackage{tkz-graph}
%\usetikzlibrary{arrows}
%\usepackage{algorithm}
%\usepackage{algorithmic}

\lstset{language=C, tabsize=4,framexleftmargin=5mm,breaklines=true}

\begin{document}

% ·············· ENCABEZADO - PIE DE PAGINA ············
\pagestyle{fancy}
\fancyhf{}
\lhead{\textbf{MA5204: Aprendizaje de Máquinas 2020}}
\rfoot{Page \thepage}
%Encabezado

% =============== Inicio Documento ===============
%\rm
\headheight = 14pt
\begin{center}
\large {\textbf{Tarea 1}}\\
\end{center}
\textbf{Profesor:} Felipe Tobar\\ 
\textbf{Auxiliares:} José Díaz, Diego Garrido, Jou-Hui Ho, Luis Muñoz, Diego Troncoso \\
\textbf{Consultas:} Diego Garrido,  Diego R. Troncoso  (U-cursos)\\
\textbf{Período:} 13/4/2020 --- 20/4/2020 \\

\noindent\textbf{Formato entrega:} Informe en formato PDF, con una extensión máxima de 3 páginas (puede usar un formato de doble columna), presentando y analizando sus resultados, y detallando la metodología utilizada. Adicionalmente debe entregar el jupyter notebook (o el código que haya generado) con la resolución de la tarea.

\vspace{5mm}

\noindent\textbf{P1. Descomposición Sesgo-Varianza} \textbf{(2.0 puntos)}
\vspace{5 mm}
\begin{itemize}
    \item[(a)] (0.5 puntos) Considere el conjunto de  observaciones  $D=\{(x_{i}, y_{i})\}_{i=1}^{N}$, relacionadas mediante el modelo lineal  
    	\begin{equation*}
    	    y_{i}=\theta^\top x_{i}+\epsilon_{i},
    	\end{equation*}
    	 donde $\{\epsilon_{i}\}_{i=1}^N$ son observaciones i.i.d con $\mathbb{E}[\epsilon]=0$ y $\text{var}(\epsilon)=\sigma^{2}$,  $\theta$ es un parámetro fijo y desconocido. Para un nuevo par, denotado $(x,y)$, no contenido en el conjunto de observaciones $D$, considere la predicción de $y$ mediante $\hat y=\hat\theta^\top x$, donde $\hat{\theta}$ es un estimador del  parámetro $\theta$ basado en $D$. Muestre que el costo cuadrático esperado de predecir $y$  con $\hat y$ (recuerde que las esperanzas se toman con   respecto a la ley de $\epsilon$) admite la siguiente  descomposición sesgo-varianza:
    	\begin{equation*}
    	    \mathbb{E}\left[(\hat{y}-y)^{2}\right] = \text{var}(\hat{y})+\text{sesgo}^{2}(\hat{y}) + \sigma^{2},
    	\end{equation*}
        donde: i) el primer término es la varianza de la variable aleatoria $\hat y$, con $\text{var}(\hat{y})=\mathbb{E}\left[(\hat{y}-\mathbb{E}[\hat{y}])^{2}\right]$ (recuerde que $\hat\theta$ es variable aleatoria, pues depende de $D$), ii) el segundo término es el sesgo al cuadrado de la variable aleatoria $\hat{y}$, con $\text{sesgo}(\hat{y})=E\left[\hat{y}-y\right]=E\left[\hat{y}\right]- \theta^\top x$, y  iii) $\sigma^2$ es la varianza de $\epsilon$.

% 	\item[(a)] (0.5 puntos) Considere el conjunto de  observaciones  $D=\{(x_{i}, y_{i})\}_{i=1}^{N}$, relacionados mediante el modelo lineal  
% 	\begin{equation}
% 	    y_{i}=\theta^\top x_{i}+\epsilon_{i},
% 	\end{equation}
% 	 donde $\{\epsilon_{i}\}_{i=1}^N$ son observaciones i.i.d con $\text{var}(\epsilon)=\sigma^{2}$, y $\theta$ es un parámetro fijo y desconocido. Para un nuevo par, denotado $(x,y)$, no contenido en el conjunto de observaciones $D$, considere la predicción de $y$ mediante $\hat f=\hat\theta^\top x$, donde $\hat{\theta}$ es un estimador del  parámetro $\theta$ basado en $D$. Muestre que el costo cuadrático esperado de predecir $y$  con $\hat f$ (recuerde que las esperanzas se toman con   respecto a la ley de $\epsilon$) admite la siguiente  descomposición sesgo-varianza:
% 	\begin{equation*}
% 	    \mathbb{E}\left[(\hat{f}-y)^{2}\right] = \text{var}(\hat{y})+\text{sesgo}^{2}(\hat{y}) + \sigma^{2},
% 	\end{equation*}
%     donde: i) el primer término es la varianza de la variable aleatoria (VA) $\hat y$ (recuerde que $\hat\theta$ es VA, pues depende de $D$), ii) el segundo término es la diferencia de la esperanza de $\hat y$ y el valor real de la parte determinista de $y$, dado por $f=\theta^\top x$, y  iii) $\sigma^2$ es la varianza de $\epsilon$.
	\item[(b)] (1.0 puntos) Para los parámetros de mínimos cuadrados $\theta_{MC} = (X^\top X) ^{-1}X^\top Y $  y de  mínimos cuadrados regularizados $\theta_{MCR} = (X^\top X + \rho I) ^{-1}X^\top Y$ calcule el sesgo y la varianza de la predicción.
	\item[(c)] (1.0 puntos) Analice las expresiones anteriores. ¿Cuáles son las ventajas y desventajas de ambos estimadores? ¿Qué aseveraciones  puede hacer de la comparación de ambos criterios (MC y MCR) con respecto de  sus  sesgos y  varianzas?
	
\end{itemize}

\vspace{5 mm}
\noindent\textbf{P2. Regresión Lineal} \textbf{(4.0 puntos)}
\vspace{5 mm}

Como primer paso debe instalar \href{https://www.anaconda.com/distribution/}{Anaconda} una distribución de Python que proporciona el stack básico de Python para la ciencia de datos, debe descargar la versión Python 3.7. Anaconda incluye  \href{https://jupyter.org/}{Jupyter Notebook}, un entorno de desarrollo interactivo web.\\

Para esta pregunta se pide implementar el estimador de mínimos cuadrados y de mínimos cuadrados regularizados con regularizador \textit{ridge} para un conjunto de datos. Para la implementación de los estimadores \textbf{solo está permitido el uso de operaciones de álgebra lineal}, para esto pueden utilizar el stack de numpy.\\

Como base de datos del experimento se utilizará el archivo \textbf{Housing.csv}, este corresponde a un dataset que posee precios de casas de algunas localidades de USA. Este dataset consta de  una variable $X$, que corresponde al ingreso promedio de la población en esa área (\textit{Avg Area Income}), e $Y$, que corresponde al precio de las casas. El objetivo de esta pregunta es aprender una función lineal que relacione $X$ e $Y$, así un agente inmobiliario que no tiene información sobre el precio de las casas en una nueva localidad pueda fijarle un precio a estas en base al ingreso promedio del área. Para una correcta comparación de los estimadores la base de datos viene dividida en dos conjuntos, entrenamiento (\textit{in-sample}) y validación (\textit{out-of-sample}).

Para esto deberá:
\begin{itemize}
	\item[(a)] (0.5 puntos) Cargar los datos desde el archivo \textbf{Housing.csv} y graficarlos.
	\item[(b)] (1.0 puntos) Implemente el estimador de mínimos cuadrados regularizados usando regularización \textit{ridge} y obtenga el vector de parámetros $\theta$ para diferentes valores de $\rho \in [0, 10]$ incluyendo los límites. Para esto implemente la función \texttt{reg\_lineal(X,Y,rho)}. 
	\item[(c)] (0.5 puntos) Grafique el valor de los parámetros estimados para los valores de $\rho$.
	\item[(d)] (0.5 puntos) Grafique el error cuadrático medio y la varianza de la predicción para los valores de $\rho$ tanto en el conjunto de entrenamiento como en el de validación. 
	\item[(e)] (0.5 puntos) Grafique la ecuación de la recta con los parámetros estimados para diferentes valores de $\rho$ junto a los datos de entrenamiento y validación.
	\item[(f)] (1.0) Discuta cómo elegir  el valor apropiado de $\rho$ en base a los resultados obtenidos en los puntos anteriores.
\end{itemize}

 \textbf{No} se permite el uso de paquetes predefinidos para regresión lineal. Estos pueden ser considerados para contrastar los propios resultados pero no para resolver la pregunta. E.g., \texttt{numpy.polyfit},  \texttt{scipy.stats.linregress}, \texttt{sklearn.linear\_model.LinearRegression}.

\end{document}


