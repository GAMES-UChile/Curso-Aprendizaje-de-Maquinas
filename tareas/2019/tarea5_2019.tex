\documentclass[11pt,letterpaper]{article}
\usepackage[utf8]{inputenc}
\usepackage[letterpaper,includeheadfoot, top=0.5cm, bottom=3.0cm, right=1.5cm, left=1.5cm]{geometry}
\renewcommand{\familydefault}{\sfdefault}
\usepackage{float} % Allows putting an [H] in \begin{figure} to specify the exact location of the figure
\renewcommand{\figurename}{Fig.}
\usepackage{amsmath}
\usepackage{graphicx}
\usepackage{color}
\usepackage{hyperref}
\usepackage{amssymb}
\usepackage{url}
\usepackage{pdfpages}
\usepackage{fancyhdr}
\usepackage{subfig}
\usepackage{listings} %Codigo
\usepackage{selinput}                   % Compatibilidad con acentos
\newcommand{\bR}{\ensuremath{\mathbb{R}}}
\newcommand{\bN}{\ensuremath{\mathbb{N}}}
\newcommand{\bZ}{\ensuremath{\mathbb{Z}}}
\newcommand{\bP}{\ensuremath{\mathbb{P}}}
\newcommand{\bE}{\ensuremath{\mathbb{E}}}
\newcommand{\bD}{\ensuremath{\mathbb{D}}}
\newcommand{\bV}{\ensuremath{\mathbb{V}}}
\newcommand{\cN}{\ensuremath{\mathcal{N}}}
\newcommand{\x}{\ensuremath{\mathbf{x}}}
\newcommand{\m}{\ensuremath{\mathbf{m}}}
\newboolean{pauta}
\setboolean{pauta}{false}
%\usepackage{tkz-graph}
%\usetikzlibrary{arrows}
%\usepackage{algorithm}
%\usepackage{algorithmic}

\lstset{language=C, tabsize=4,framexleftmargin=5mm,breaklines=true}

\begin{document}

% ·············· ENCABEZADO - PIE DE PAGINA ············
\pagestyle{fancy}
\fancyhf{}
\lhead{\textbf{MA5203: Aprendizaje de Máquinas Probabilístico 2019}}
\rfoot{Page \thepage}
%Encabezado

% =============== Inicio Documento ===============
%\rm
%\headheight = 35.02448pt
\begin{center}
\large {\textbf{Tarea 5: Redes Neuronales}}\\
\end{center}
\textbf{Profesor:} Felipe Tobar\\ 
\textbf{Auxiliares:} Mauricio Araneda, Alejandro Cuevas y Mauricio Romero \\
\textbf{Consultas:} Todo el cuerpo docente.\\
\textbf{Fecha entrega:} 23/6/2019 \\
\textbf{Formato entrega:} Entregue un informe en formato PDF con una extensión de a lo más 2 páginas presentando y analizando sus resultados, detalle la metodología utilizada y adicionalmente debe entregar un jupyter notebook con los códigos que creó para resolver la tarea.

\begin{center}
    \textit{Esta es una de las 2 versiones de la tarea4, deben elegir solo una de las versiones (SVM o redes).}
\end{center}

\noindent\textbf{P1. Clasificación con redes neuronales}
\vspace{5 mm}

Se pretende realizar la tarea de clasificación sobre el dataset MNIST, usando Redes neuronales como clasificador, para esto:
    
\begin{enumerate}

\item[a)](1 pts.) Cargue los datos (Vea el demo para cargarlos) y sepárelos de forma aleatoria en conjuntos de entrenamiento $5/7$, validación $1/7$ y prueba $1/7$ (si carga con pytorch el conjunto de prueba ya esta definido).

Normalice los datos de forma que el valor de cada pixel se encuentre en el rango $[0, 1]$ en vez de $[0, 255]$, para esto divida todo el dataset por $255$. 
    
\textit{En caso de cargar los datos con Pytorch:} Transforme sus datos de tal forma que cada imagen de $28 \times 28$ sea un vector de tamaño $784$.

\textbf{Observación:} Le puede ser util train\_test\_split de sklearn.

\item[b)](3 pts.) Investigue al menos 3 algoritmos de optimización utilizados para el entrenamiento de redes neuronales y compárelos con gradiente descendente vainilla en términos algorítmicos, hiperparámetros, etc. Seleccione un algoritmo para utilizarlo en su tarea y fundamente su elección.

\item[c)](3 pts.) Construya una clase MLP que le permita agregar capas y regularizadores en el constructor. Es decir, al instanciar la clase debe poder ingresar número de capas, neuronas por capas y opciones extras como regularización o dropout.

\item[d)](5 pts.) Entrene y pruebe las siguientes arquitecturas:
\begin{enumerate}
    \item[1.] MLP 1 capa.
    \item[2.] MLP 2 capas.
    \item[3.] MLP 1 capa con regularización L2.
    \item[4.] MLP 1 capa con reg. dropout en capa oculta.
\end{enumerate}

Comente lo que sucede durante el entrenamiento con la función de costo, muestre como cambia en función de las épocas y analice el resultado de clasificación en base a la matriz de confusión.

¿Qué puede decir sobre la regularización en cada caso? ¿Cómo afecta el número de parámetros de los modelos bajo estudio?

Utilizando el conjunto de validación guarde$^{\star}$ el mejor modelo obtenido durante el entrenamiento y compare las métricas de desempeño en el conjunto de test, en base al mejor modelo en cada caso.

\textbf{Observaciones}: En cuanto a los demás parámetros de red, tales como función de activación, cantidad de neuronas por cada y tazas de aprendizaje se recomienda:

\begin{itemize}
    \item Utilice alrededor de $100$ neuronas por capa.
    \item Entrene un número cercano a $20$ épocas.
    \item Se recomienda utilizar funciones de activación ReLu.
    \item Dada esa función de activación, se romienda una tasa de aprendizaje del orden de $lr=2\cdot 10^{-4}$. (Tomar en cuenta que con otra función de activación puede variar el orden de magnitud de la tasa a usar)
    \item Se recomienda un tamaño de batch $batch_{size}=100$.
    \item La regularización L2 se puede agregar en el optimizador a través del argumento \textit{weight\_decay}. 
    \item $\star$ Para guardar y cargar modelos en Pytorch revise \url{https://pytorch.org/tutorials/beginner/saving_loading_models.html} .
\end{itemize}


\end{enumerate}


\end{document}

